\documentclass{article}
\usepackage{spconf,amsmath,amssymb,graphicx,ngerman,bm}
\selectlanguage{USenglish}

% Example definitions.
% --------------------
\def \x{{\mathbf x}}
\def \v{{\mathbf v}}
\def \w{{\mathbf w}}
\def \M{{\mathbf M}}
\def \m{{\bm \mu}}
\def \k{{\mathbf \Sigma}}
\def \nv{{\mathcal N}}
\def \L{{\cal L}}

% Title.
% ------
\title{REVISITING SEMI-CONTINUOUS HIDDEN MARKOV MODELS}

%
% Single address.
% ---------------
\name{
K.~Riedhammer$^1$,
T.~Bocklet$^1$,
A.~Ghoshal$^{2,3}$,
D.~Povey$^4$
}
\address{
$^1$Pattern Recognition Lab, University of Erlangen-Nuremberg, \sc{Germany}\\
$^2$Spoken Language Systems, Saarland University, \sc{Germany}\\
$^3$Centre for Speech Technology Research, University of Edinburgh, UK\\
$^4$Microsoft Research, Redmond, WA, {\sc USA}\\
{\small\tt korbinian.riedhammer@informatik.uni-erlangen.de}
}

\begin{document}
\ninept

\maketitle

\begin{abstract}
In the past decade, semi-continuous hidden Markov models (SC-HMMs) have not 
attracted much attention in the speech recognition community. Growing amounts 
of training data and increasing sophistication of model estimation led to the 
impression that continuous HMMs are the best choice of acoustic model.
%
However, recent work on recognition of under-resourced languages faces the same
old problem of estimating a large number of parameters from limited amounts 
of transcribed speech.
This has led to a renewed interest in methods of reducing the number of parameters 
while maintaining or extending the modeling capabilities of continuous models.
%
In this work, we compare continuous, classic and multiple-codebook semi-continuous,
with full covariance matrices, and subspace Gaussian mixture models.
%
Experiments using on the RM and WSJ corpora show that a semi-continuous system
can still yield competitive results while using fewer Gaussian components.
\end{abstract}

\begin{keywords}
automatic speech recognition, acoustic modeling
\end{keywords}

\section{Introduction}
\label{sec:intro}
In the past years, semi-continuous hidden Markov models (SC-HMMs) \cite{huang1989shm}
have not attracted much attention in the automatic speech recognition (ASR) community. 
Most major frameworks (e.g.~CMU {\sc Sphinx} or HTK) more or less retired
semi-continuous models in favor of continuous models, which performed 
considerably better with the growing amount of available training data.
%
However, topics like cross-lingual model initialization and model training
for languages with little training data became more and more popular recently,
along with efforts on how to reduce the number of parameters of a continuous
system while maintaining or even extending the modeling capabilities and thus
the robustness and performance.
%
Typical techniques to reduce the number of parameters for continuous
systems include generalizations of the phonetic context modeling to reduce the 
number of states or state-tying to reduce the number of Gaussians.

It is well known that context dependent states significantly improve the
recognition performance (e.g.,~from monophone to triphone). However, 
increasing the number of states of a continuous system not only requires more 
Gaussians to be estimated, but also implies that the training data for each state 
is reduced.
%
This is where semi-continuous models hold an advantage: the only state specific 
variables are the weights of the Gaussians, while the codebook Gaussians 
themselves are always estimated on the whole data set. 
In other words, increasing the number of states leads to only modest increase 
in the total number of parameters. The Gaussian means and variances are estimated 
using all of the available data, making it possible to robustly estimate even full 
covariance matrices.
%
This also allows to reliably estimate arbitrarily large phonetic context 
(``polyphones''), which has been shown to achieve good performance for the 
recognition of spontaneous  speech \cite{schukattalamazzini1994srf,schukattalamazzini1995as} 
while having a relatively small number of parameters.
%
Further advantages of using codebook Gaussians are that they need to be 
evaluated only once per frame, and that the codebook can be initialized, trained
and adapted using even untranscribed speech.

Semi-continuous models can be extended to multiple codebooks; by
assigning certain groups of states to specific codebooks, one can find a
hybrid between a continuous and semi-continuous models that combine the strength
of state-specific Gaussians with reliable parameter estimation \cite{prasad2004t2b}.

Extending the idea of sharing parameters, subspace Gaussian mixture models
\cite{povey2011sgm} assign state specific weights and means, while the latter
are derived from the means of a shared codebook (universal background model, UBM)
using a state, phone or speaker specific transformation, thus limiting the state
Gaussians to a subspace of the UBM.

% TODO Fix the number of hours for RM and WSJ SI-284
In this article, we experiment with classic SC-HMMs and two-level tree based 
multiple-codebook SC-HMMs which are, in terms of parameter sharing, somewhere in 
between continuous and subspace Gaussian mixture models; 
we compare the above four types of acoustic models on a small (Resource Management, RM, 5 hours)
and medium sized corpus (Wall Street Journal, WSJ, 17 hours) of read English while keeping acoustic 
frontend, training and decoding unchanged. 
The software is part of the {\sc Kaldi} speech recognition toolkit 
\cite{povey2011tks} and is freely available for download, along with the
example scripts to reproduce the presented experiments.


\section{Acoustic Modeling}
\label{sec:am}
In this section, we summarize the different forms of acoustic model we use: 
the continuous, semi-continuous, multiple-codebook semi-continuous, and 
subspace forms of Gaussian Mixture Models. 
We also describe the phonetic decision-tree building process, which is 
necessary background for how we build the multiple-codebook systems.

\subsection{Acoustic-Phonetic Decision Tree Building}
The acoustic-phonetic decision tree provides the link between phones
in context and emission probability density functions (pdfs). 
%
The statistics for the tree clustering consist of the 
sufficient statistics needed to estimate a Gaussian for each seen combination 
of (phonetic context, HMM-state).  These statistics are based on a Viterbi 
alignment of a previously built system.  The roots
of the decision can be configured; for the RM experiments, these correspond to
the phones in the system, and on Wall Street Journal, they
correspond to the ``real phones'', i.e. grouping different stress and
position-marked versions of the same phone together.  

The splitting procedure can ask questions not only about the context phones,
but also about the central phone and the HMM state; the phonetic questions are 
derived from an automatic clustering procedure.  The splitting procedure is
greedy and optimizes the likelihood of the data given a single Gaussian in each
tree leaf; this is subject to a fixed variance floor to avoid problems caused
by singleton counts and numerical roundoff.  We typically split up to a 
specified number of leaves and then cluster the resulting leaves as long as the
likelihood loss due to combining them is less than the likelihood improvement
on the final split of the tree-building process; we typically restrict this 
clustering to avoid combining leaves across different subtrees (e.g., from 
different monophones).  The post-clustering process typically reduces the number 
of leaves by around 20\%.

\subsection{Continuous Models}
For continuous models, every leaf of the tree is assigned an separate
Gaussian mixture model. Formally, the emission probability of tree leaf $j$ is 
computed as
\begin{equation}
p(\x | j) = \sum_{i=1}^{N_j} c_{ji} \nv(\x; \m_{ji}, \k_{ji}) 
\end{equation}
where $N_j$ is the number of Gaussians assigned to $j$, and the $\m_{ji}$ and
$\k_{ji}$ are the means and covariance matrices of the mixtures.

We initialize the mixtures with a single component each, and subsequently 
allocate more components by splitting components at every iteration until a 
pre-determined number of components is reached (in this case 9000). We allocate 
the Gaussians proportional to a small power (e.g.~0.2) of the state's 
occupation count.

\subsection{Semi-continuous Models}
The idea of semi-continuous models is to have a large number of Gaussians that
are shared by every tree leaf using individual weights, thus the emission 
probability of tree leaf $j$ can be computed as
\begin{equation}
p(\x | j) = \sum_{i=1}^{N} c_{ji} \nv(\x; \m_i, \k_i) 
\end{equation}
where $c_{ji}$ is the weight of mixture $i$ for leaf $j$. As the means
and covariances are shared, increasing the number of states only implies
a small increase in number of parameters even for full-covariance Gaussians.
Furthermore, the Gaussians need to be evaluated only once for each $\x$.

Another advantage is the initialization and use of the codebook. It can be
initialized and adapted in a fully unsupervised manner using expectation
maximization (EM), maximum a-posteriori (MAP), maximum likelihood linear
regression (MLLR) and similar algorithms on untranscribed audio data.

For better performance, we initialize the codebook using the tree statistics
collected on a prior phone alignment. For each tree leaf, we include
the respective Gaussian as a component in the mixture.
The components are merged and (if necessary) split to eliminate low count 
Gaussians and to match the desired number of components. In a final step, 
5 EM iterations are used to improve the goodness of fit to the acoustic 
training data.

\subsection{Two-level Tree based Multiple-Codebook Semi-continuous Models}
A style of system that lies between the semi-continuous and continuous types of
systems is one based on a two-level tree; this is described in~\cite{prasad2004t2b}
as a state-clustered tied-mixture (SCTM) system. The way we implement this is
to first build the decision tree used for phonetic clustering to a relatively
coarse level (e.g.~100); and to then split more finely (e.g.~2500), but for
each new leaf, remember which leaf of the original tree it corresponds to.
Each leaf of the first tree corresponds to a codebook of Gaussians, i.e.~the 
Gaussian parameters are tied at this level.  For this type of system we do not
apply any post-clustering.
The leaves of the second tree contain the weights, and these leaves correspond 
to the actual context-dependent HMM states.  

The way we formalize this is that the context-dependent states $j$ each have 
a corresponding codebook indexed $k$, and the function $m(j) \rightarrow k$ 
maps from the leaf index to the codebook index. The likelihood function is now
%
\begin{equation}
p(\x | j) = \sum_{i=1}^{N_{m(j)}} c_{ji} \, \nv\left(\x; \m_{m(j), i}, \k_{m(j), i}\right)
\end{equation}
%
where $\m_{m(j), i}$ is the $i$-th mean vector of the codebook
associated with $j$.

Similar to the traditional semi-continuous codebook initialization, we start
from the tree statistics and distribute the Gaussians to the respective codebooks
using the tree mapping to obtain preliminary codebooks.
The target size $N_k$ of codebook $k$ is determined with respect to a power
of the occupancies of the respective leaves as
\begin{equation}
N_k = N_0 + \frac
%% power rule, maybe some other day...
%  { \left(  \sum_{l \in \{m(l) = k\}} \text{occ}(l)  \right)^q }
%  { \sum_r \left(  \sum_{t \in \{m(t) = r\}} \text{occ}(l)  \right)^q } 
  { \sum_{l \in \{m(l) = k\}} \text{occ}(l) }
  { \sum_r \sum_{t \in \{m(t) = r\}} \text{occ}(l) } 
  \left( N - K \cdot N_0 \right)
\end{equation}
where $N_0$ is a minimum number of Gaussians per codebooks (e.g., 3), $N$ is
the total number of Gaussians, $K$ the number of codebooks, and $\text{occ}(j)$ 
is the occupancy of tree leaf $j$. 
%
%% Korbinian: No power rule for now, seems to harm results.
%, and $q$ controls the influence of the occupancy regarding one codebook. 
%A typical value of $q=0.2$ leads to rather homogeneous codebook sizes while 
%still attributing more Gaussians to states with larger occupancies. 
%For the two-level architecture, a $q=1$ yielded best results.
%
The target sizes of the codebooks are again enforced by either splitting or
merging the components.

\subsection{Subspace Gaussian Mixture Models}
The idea of subspace Gaussian mixture models (SGMM) is, similar to 
semi-continuous models, to reduce the number of parameters by selecting the
Gaussians from a subspace spanned by a universal background model (UBM)
and state specific transformations. In principle, the SGMM emission pdfs can
be computed as
\begin{eqnarray}
p(\x | j) & = & \sum_{i=1}^{N} c_{ji} \nv(\x; \m_{ji}, \k_i) \\
\m_{ji}    & = & \M_i \v_j \\
c_{ji}     & = & \frac{\exp \w_i^T \v_j}{\sum_l^N \exp \w_l^T \v_j}
\end{eqnarray}
where the covariance matrices $\k_i$ are shared between all leafs $j$. The
weights $w_{ji}$ and means $\m_{ji}$ are derived from $\v_j$ together with
$\M_i$ and $\w_i$. The term ``subspace'' indicates that the parameters of the
mixtures are limited to a subspace of the entire space of parameters of
the underlying codebook.
%
A detailed description and derivation of the accumulation and update formulas
can be found in \cite{povey2011sgm}. Note that the experiments in this article
are without adaptation.

\section{Smoothing Techniques for Semi-continuous Models}
%
\subsection{Intra-Iteration Smoothing}
Although an increasing number of tree leaves does not imply a huge increase in
parameters, the sufficient statistics for the individual leaves will be
sparser, especially for little training data.
The {\em intra-iteration} smoothing is motivated by the fact that tree leaves
that share the root are similar, thus, if the sufficient statistics of the
weights of a leaf $j$ are very small, they should be interpolated with the 
statistics of closely related leaf nodes.
To do so, we first propagate all sufficient statistics up the tree so that the
statistics of any node is the sum of its children's statistics.
Second, the statistics of each node and leaf are interpolated top-down with 
their parent's using an interpolation weight $\rho$ in
\begin{equation} \label{eq:intra}
\hat\gamma_{ji} \leftarrow 
  \underbrace{\left(\gamma_{ji} + \frac{\rho}{\left( \sum_k \gamma_{p(j),k} \right) + \epsilon} \, \gamma_{p(j),i}\right)}_{\mathrel{\mathop{:}}= \bar\gamma_{ji}}
  \cdot 
  \underbrace{\frac{\sum_k \gamma_{jk}}{\sum_k \bar\gamma_{jk}}}_\text{normalization}
\end{equation}
where $\gamma_{ji}$ is the accumulator of weight $i$ in tree leaf $j$
and $p(j)$ is the mapping to the parent node of $j$.
%
For two-level tree based models, the propagation and interpolation cannot
be applied across different codebooks. 
%
A similar yet different interpolation scheme without normalization was used in 
\cite{schukattalamazzini1994srf,schukattalamazzini1995as} along with a different
tree structure.

%This is similar to the {\em I} step of the {\em APIS} algorithm 
%\cite{schukattalamazzini1994srf,schukattalamazzini1995as} but smoothes the
%statistics with respect to their counts instead of replacing them.
%This modification is necessary as the tree in \cite{schukattalamazzini1995as}
%is constructed in a way, that the tree splits (nodes) represent acoustic
%models (the {\em polyphones}), i.e., the first node level would be the
%monophones, and any phone with more context would be a child node, with only
%the phones with the largest context being a leaf. For the training, any data contributing
%to a node $j$ should also contribute to its parent $p(j)$, and if a child
%has insufficient statistics, these can be interpolated with the parent's. 
%As we attach models only to the leaves, this motivation does not (necessarily)
%hold, thus requiring the above modification of the interpolation scheme.

\subsection{Inter-Iteration Smoothing}
Another typical problem that arises when training semi-continuous models is
that the weights tend to converge on a local optimum over the iterations. 
This is due to the fact that the codebook parameters change rather slowly 
as the statistics are collected from all respective tree leaves, but the 
leaf specific weights converge much faster.
%
To compensate for this, we smooth the newly estimated {\em parameters} 
$\Theta^{(i+1)}$ with the ones from the prior iteration using an interpolation
weight $\varrho$
\begin{equation}
\hat\Theta^{(i+1)} \leftarrow (1 - \varrho) \, \Theta^{(i+1)} + \varrho \, \Theta^{(i)} % \quad .
\end{equation}
which leads to an exponentially decreasing impact of the initial parameter set.

%----%<------------------------------------------------------------------------

\section{Systems Description}
The experiments presented in this article are computed on the DARPA Resource 
Management Continuous Speech Corpus (RM) \cite{price1993rm} and the Wall Street
Journal Corpus \cite{garofalo2007wsj}.

For both data sets, we first train an initial monophone continuous system
with about 1000 diagonal Gaussians (122 states) on a small data subset; with 
the final alignments, we train an initial triphone continuous system with 
about 10000 Gaussians (1600 states).  The alignments of this triphone system
are used as a basis for the actual system training.

Note that this baseline may be computed on any other (reduced) data set
and has the sole purpose of providing better-than-linear initial alignments
to speed up the convergence of the models.
%
The details of the model training can also be found in the {\sc Kaldi} recipes.

On the acoustic frontend, we extract 13 Mel-frequency cepstral coefficients,
compute deltas and delta-deltas, and apply cepstral mean normalization. For
the decoding, we use a regular 3-gram estimated using the IRSTLM toolkit 
\cite{federico2008iao}.

\subsection{Resource Management}
For the RM data set, we train on the speaker independent training and development 
set (about 4000 utterances) and test on the six DARPA test runs Mar and Oct '87,
Feb and Oct '89, Feb '91 and Sep '92 (about 1500 utterances total).  The parameters 
were tuned to the test set.
%
\begin{itemize}
\item
  {\em cont}: continuous triphone system using 9011 diagonal covariance
  Gaussians in 1480 tree leaves
\item
  {\em semi}: semi-continuous triphone system using 768 full covariance
  Gaussians in 2500 tree leaves, no smoothing (explanations later).
\item
  {\em 2lvl}: two-level tree based semi-continuous triphone system using
  3072 full covariance Gaussians in 208 codebooks ($4/14.7/141$ min/average/max
  components), 2500 tree leaves, $\rho = 35$ and $\varrho = 0.2$.
\item
  {\em sgmm}: subspace Gaussian mixture model triphone system using a 400 
  full-covariance Gaussian background model and 2016 tree leaves.
\end{itemize}

\subsection{WSJ}
% TODO fix the WSJ train/dev/eval description
For the WSJ data set, we trained on the SI-284 training set, tuned on dev
and tested on eval92,93.
As the classic semi-continuous system did not yield good performance on the
RM data both in terms of run-time and performance, we omit it for the WSJ
experiments.
%
% TODO Korbinian: There are more expts ongoing, they should be done by 
%                 tomorrow! I will also fill in the details below then.
\begin{itemize}
\item
  {\em cont}: continuous triphone system using 10000 diagonal covariance
  Gaussians in 1576 tree leaves.
%\item
%  {\em semi}: semi-continuous triphone system using 768 full covariance
%  Gaussians in 2500 tree leaves, but no smoothing (explanations below).
\item
  {\em 2lvl}: two-level tree based semi-continuous triphone system using
  4096 full covariance Gaussians in 208 codebooks ( min/average/max
  components),  tree leaves, $\rho = $ and $\varrho = $.
\item
  {\em sgmm}: subspace Gaussian mixture model triphone system using a 400 
  full-covariance Gaussian background model and 2361 tree leaves.
\end{itemize}

%----%<------------------------------------------------------------------------

\section{Results}

%------%<----------------------------------------------------------------------
\subsection{RM}

\begin{table}%[tb]
\begin{center}
%\footnotesize
%\begin{tabular}{|l||r|r|r|r|r|r||c|}
\begin{tabular}{|l||c|c|c|c|c|c||c|}
\hline
~               & ma87 & oc87 & fe89 & oc89 & fe91 & se92 & {\em avg}  \\ \hline\hline
%{\em mono/cont} &  6.24 &  8.72 &  7.85 &  9.80 &  8.21 & 13.60 & 9.50 \\ \hline
%{\em tri1/cont} &  0.96 &  2.76 &  2.69 &  3.61 &  3.30 &  6.33 & 3.64 \\ \hline\hline
{\em cont} &  1.08 &  2.48 &  2.69 &  3.46 &  2.66 &  5.90 & 3.38 \\ \hline
{\em semi} &  1.80 &  3.19 &  4.72 &  4.62 &  4.15 &  6.88 & 4.66 \\ \hline 
{\em 2lvl} &  0.48 &  1.70 &  2.46 &  3.35 &  1.89 &  5.31 & {\bf 2.90} \\ \hline
{\em sgmm} &  0.48 &  2.20 &  2.62 &  2.50 &  1.93 &  5.12 & 2.78 \\ \hline
\end{tabular}
\end{center}
\caption{\label{tab:res_rm}
Detailed recognition results in \% WER for the six DARPA test sets using 
different acoustic models on the RM data set.
}
\end{table}

The results on the different test sets on the RM data are displayed in 
Tab.~\ref{tab:res_rm}. The classic semi-continuous system shows the worst 
performance of the four. Possible explanations are an insufficient number
of Gaussians paired overfitting of the weights to a little number of 
components. Unfortunately, this could not be eased with the interpolations
and larger numbers of Gaussians would unacceptably increase the computational 
load due to the full covariance matrices.

The two-level tree based multiple-codebook system performance lies within the 
continuous and SGMM system, which goes along with the modeling capabilities 
and the rather little training data. 

%------%<----------------------------------------------------------------------
% Now let's talk about the diag/full cov and interpolations

\begin{table}%[tb]
\begin{center}
\begin{tabular}{|c|c||c|c|c|c|}
\hline
covariance & Gaussians & none & inter & intra & both \\ \hline\hline
full       &      1024 & 3.70 & 3.64 & 3.79 & 3.74 \\ \hline
full       &      3072 & 3.01 & 3.01 & 3.02 & 2.90 \\ \hline\hline
diagonal   &      3072 & 4.13 & 4.15 & 4.25 & 4.35 \\ \hline
diagonal   &      9216 & 3.22 & 3.09 & 3.28 & 3.20 \\ \hline
\end{tabular}
\end{center}
\caption{\label{tab:rm_diagfull}
Average \% WER of the multiple-codebook semi-continuous model using different 
numbers diagonal and full covariance Gaussians, and different smoothings 
on the RM data. 
Settings are 208 codebooks, 2500 context dependent states; $\rho = 35$ 
and $\varrho = 0.2$ if active.
}
\end{table}

Keeping the number of codebooks and leaves as well as the smoothing parameters
$rho$ and $\varrho$ of {\em 2lvl} constant, we experiment with the number of
Gaussians and type of covariance; the results are displayed in 
Tab.~\ref{tab:rm_diagfull}.
Interestingly, the full covariances make a strong difference: Using the same
number of Gaussians, full covariances lead to a significant improvement.
On the other hand, a substantial increase of the number of diagonal Gaussians 
leads to a similar performance as the regular continuous system.
%
Another observation from Tab.~\ref{tab:rm_diagfull} is that the smoothing
parameters need to be carefully calibrated. While the inter-iteration smoothing
helps in most cases, the intra-iteration smoothing coefficient $\rho$ strongly
depends on the type and number of Gaussians, which is due to the direct influence
of $\rho$ on the counts in Eq.~\ref{eq:intra}.

\subsection{WSJ}

% TODO Korbinian: More expts are ongoing; these include the full training set
%                 and cmvn, and hopefully a good param combination for 2lvl...
\begin{table}%[tb]
\begin{center}
\begin{tabular}{|l||c|c|c|c|c|c||c|}
\hline
~          &    dev & eval92 & eval93 & {\em avg} \\ \hline\hline
{\em cont} &        &  13.17 &  18.61 &     15.23 \\ \hline
% Korbinian: no semi-continuous for WSJ, takes too long to compute; will do it
%            for the journal, though.
%{\em semi} &        &        &        &           \\ \hline 

% tri2-2lvl-208-3072-4000-0-0-0-0 !! no cmvn !!
%{\em 2lvl} &  13.95 &  21.61 &     16.85 \\ \hline

% tri2-2lvl-208-4096-6000-0-1-35-0.2 !! no cmvn !!
{\em 2lvl} &        &  12.83 &  21.61 &     16.16 \\ \hline

{\em sgmm} &        &  10.76 &  17.82 &     13.44 \\ \hline
\end{tabular}
\end{center}
\caption{\label{tab:res_wsj}
Detailed recognition results in \% WER for the '92 and '93 test sets using 
different acoustic models on the WSJ data.
}
\end{table}

% TODO Korbinian: This needs some more text once the above table is filled
The results on the different test sets on the WSJ data are displayed in
Tab.~\ref{tab:res_rm}.

\section{Summary}
In this article, we compared continuous models and SGMM to 
two types of semi-continuous hidden Markov models, one using a single codebook 
and the other using multiple codebooks based on a two-level phonetic decision tree.
%
While the first could not produce convincing results, the multiple-codebook 
architecture shows promising performance, especially for limited training data. 

Although the current performance is below the state-of-the-art, the
rather simple theory and low computational complexity, paired with the possibility
of solely acoustic adaptation make two-level tree based semi-continuous acoustic
models an attractive alternative to low-resource applications -- both in terms
of computational power and training data.
%
The take home message from the experiments is that the choice of acoustic model
should be made based on a resource constraint (number of Gaussians, available
compute power), and that both continuous and semi-continuous models can do the job.

For future work, effect of adaptation and especially discriminative training
with respect to the codebook architecture need to be explored.
Furthermore, the SGMM framework can be extended to two-level architecture to 
reduce computational effort in both training and decoding.

%\section{REFERENCES}
%\label{sec:ref}

% Korbinian: We might need that ;)
%\footnotesize
\bibliographystyle{IEEEbib}
\bibliography{refs-eig,refs}

\end{document}
